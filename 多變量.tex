\documentclass{beamer}
\usetheme{Singapore}
\usecolortheme{dove}
% for themes, etc.
\mode<presentation>
%\usetheme{Warsaw}
%\usecolortheme{dolphin}
%{\usetheme{Singapore}}
%{ \usetheme{lined} }
{\usetheme{Dresden}}
\usepackage{color} %字體可換顏色
\usepackage{fontspec}   %加這個就可以設定字體
\usepackage{xeCJK}       %讓中英文字體分開設置
\usepackage{graphicx} %插入jpg圖檔
\setCJKmainfont{標楷體} %設定中文為系統上的字型,而英文不去更動,使用原TeX字型
\XeTeXlinebreaklocale "zh"             %這兩行一定要加,中文才能自動換行
\XeTeXlinebreakskip = 0pt plus 1pt     %這兩行一定要加,中文才能自動換行
\usepackage{amsmath,amssymb,amsfonts,booktabs}
\usepackage{epic}
\usepackage{mathpazo}  % fonts are up to you
%\usepackage{astats,epsfig}
\usepackage{graphicx,epsfig,subfigure}
\usepackage{pdftexcmds}
\usepackage{xcolor}
\usepackage{amsthm}
\usepackage{bxpapersize}

\renewcommand{\proofname}{\ctxfk \textbf{Proof.}}
%\usepackage{array,booktabs}
% these will be used later in the title page
\title{調整傳統季節分類曆法 \protect\\ (多變量分析期末報告)}
\author{李侑瑾、陳俊翔、張浩榜、許劭廷}
\institute{國立東華大學 應用數學所統計組}
\date{2018/06/14}
\begin{document}
	\begin{frame}
	\titlepage
\end{frame}
        \begin{frame}<beamer>{Outline}
                \tableofcontents
        \end{frame}
\section{簡介}
\begin{frame}
\frametitle{前情提要}
\emph{1. } unsupervised learning\\
$\rightarrow$ 做 cycle、找出分群法\\
\emph{2. } supervised learning\\
$\rightarrow$ 根據現有春夏秋冬做分群
\end{frame}
\begin{frame}
\frametitle{(續)前情提要}
\includegraphics[width=2in]{10.jpg}\includegraphics[width=2in]{13.jpg}\\
\includegraphics[width=2in]{12.jpg}\includegraphics[width=2in]{11.jpg}
\end{frame}

\begin{frame}
\frametitle{註解頁}
\hspace*{4mm} 我們本來是有兩個計畫的,第一個是unsupervised learning: 去做cycle、找出分群法,另一個則是我們本次的報告,supervised learning: 根據現有春夏秋冬做分群當我們在做第一個計劃時出了一些問題,第一是做出的cycle跟現在的曆法幾乎一樣,而且也不是所有變數都能找出cycle,就算有找出並做迴歸方法,因為是unsupervised learning所以會有太多分法都是合理的,每個職業關心的分法都不一樣,以我們現在的能力工程過於浩大;第二是還要配合時間去分,不然倘若1.3.5月一群,2.4.6月一群,這樣根本沒有意義了。所以後來考量上訴因素就改成第二個計劃了。

\end{frame}



\begin{frame}
\frametitle{為什麼要調整?}
\emph{•} 人為曆法\\
\emph{•} 真實天氣\\
$\rightarrow$ 調整曆法以貼合天氣\\
\begin{center}
\includegraphics[width=1.5in]{5.jpg}\includegraphics[width=1.5in]{1.jpg}
\end{center}
%%春夏秋冬是照著曆法來訂定的,曆法是人訂出來的總有一天會不適用現實,觀察現今是不是該微調一下我們春夏秋冬的定義
\end{frame}

\begin{frame}
\frametitle{註解頁}
\hspace*{4mm} 現在的曆法可能跟真實天氣有所衝突,畢竟全球暖化等等天氣因素,造成幒股創下的曆法跟現在會出現些問題。我們想利用統計方法定義出符合現在的四季的曆法,是否應該調整現在的曆法。\\

\end{frame}

\begin{frame}
\frametitle{資料與分類工具}
\emph{•} 中央氣象局花蓮測站2010年到2018年天氣資料\\
$\rightarrow$氣壓、溫度、風、降水、日照\\
\emph{•} SLR\\
\emph{•} PCA\\
\emph{•} CART\\
\emph{•} AdaBoost

\end{frame}
\begin{frame}
\frametitle{該怎麼做}
以2010到2015年作為Training data\\
\emph{1. }目標: 利用 training data 區分「冬春」以及「夏秋」\\
\begin{center}
\includegraphics[width=3in]{2.jpeg}\\
\end{center}
\end{frame}

\begin{frame}
\frametitle{註解頁}
\hspace*{4mm} 利用training data 區分「冬春」以及「夏秋」,但以現有技術難以再分四季了,像是此圖是對第二個變數以及第四個變數作圖,藍色跟紅色是現今曆法的冬天跟春天,很明顯是分不開的,而很多都是很難分開的,所以我們先以大方向區隔出「冬春」以及「夏秋」就好。\\

\end{frame}

\begin{frame}
\frametitle{(續)該怎麼做}
\emph{2. }去除相關變數\\
\emph{3. }風向轉換(角度 vs 向量)\\
\includegraphics[width=2in]{3.jpeg}\includegraphics[width=2in]{4.jpeg}\\
\end{frame}

\begin{frame}
\frametitle{註解頁}
\emph{2. }去除相關變數\\
\hspace*{4mm} 用correlation檢查其相關性,如果高度相關則擇一留下,像是溫度與當日最高溫有很高的相關性,就留下溫度即可。其實在做去除變數的時候,我們有其中四個變數有高度相關,但用p-value只有一個被變數需要去除,另外三個p-value都很大,後來發現是一個叫” Multicollinearity”,的現象,所以我們用”並陳”的方式只留其一。\\
\emph{3. }風向轉換(角度 vs 向量)\\
\hspace*{4mm} 在經過第二個步驟之後,我們將原有的23個變數減少成15個,那在這15個變數中,還有兩個是有關風向的變數,我們特別將它拿出來討論,進行第三個步驟。\\
\hspace*{4mm} 因為在一般的認知中,風向是二維度的呈現,在資料中它是用角度呈現,也就是一維度的數值,所以這樣就會產生一個問題就是,當我們二維度的資料用一維度去呈現時,一定會產生某些資料的流失。以分別是1度跟359度做比較,所以他們差了兩度。實際上是非常接近的,但在數值上呈現卻差了358度。


\end{frame}

\begin{frame}
\frametitle{(續)該怎麼做}
\emph{4. }做出 weak classify function\\
\emph{5. }AdaBoost\\
$\rightarrow$ 給予各個 weak classify function權重,結合成一個決策函數。\\
\emph{6. }以2016年來評估分群績效\\
\emph{7. }觀察2017、2018年的天氣是否有需要微調\\
%%說明我們只有做冬春 vs 夏秋還有原因
%%先把高相關性的挑掉,另外把風向部分轉換成二維空間單位圓上的點,結果風向一點屁用都沒有(用圖說明),因此特別挑掉
%%再來,針對adaboost做weak classify function,讓每個人自由選擇三個weak classify function
%%用adaboost來整併weak classify function
%%2016當test data
%%2017 2018當作觀察
\end{frame}
\begin{frame}
\frametitle{註解頁}
\hspace*{4mm} 在決定使用哪些變數後,我們要開始處理這些變數,在處理變數的過程中我們會使用rpart、SLR、PCA、weak classify function這幾種方式將各組資料做分群。\\
\hspace*{4mm} 在將各組資料作完分群後,我們要透過AdaBoost的方式給予各個分組適當的權重,並且將其結合成一個決策函數,在完成決策函數後,接著我們先以2016年的資料來評估這個決策函數的分群績效,若該決策函數的績效是不錯的,那麼我們再以2017和2018年的天氣資料來做比較,決定是否現行的曆法需要因為氣候的變遷而有所異動。\\
\end{frame}
\section{實作過程}
\begin{frame}
\frametitle{weak classify function}

\begin{center}
SLR、PCA\\
\includegraphics[width=3in]{14.jpg}\\
\includegraphics[width=3in]{16.jpeg}
\end{center}

\end{frame}

\begin{frame}
\frametitle{(續)weak classify function}
\begin{center}
rpart 中的CART 所做出來的決策樹
\includegraphics[width=4in]{17.jpeg}
\end{center}
\end{frame}

\begin{frame}
\frametitle{Adaboost}
\begin{center}

\begin{tabular}{cc}
對應classify function&adaboost給予的權重\\
	h1&0.977484768\\
	  h2&2.043227987\\
	    h3&0.518198453\\
	     h4&-0.017431753\\
	      h5&-0.020348786\\
	       h6&-0.171191291\\
 h7&0.620454337\\
  h8&-0.088147612\\
   h9&-0.008179166\\
    h10&-0.512029051\\



\end{tabular}\\


\end{center}
\end{frame}
\begin{frame}
\frametitle{註解頁}
\hspace*{4mm} 接下來我們利用Adaboost來找適合這10個弱分群函數的權重,所以每一個弱分群函數都會有屬於他們自已的權重,因此我們可以形成一個決策函數以便來區分冬春跟夏秋。
\end{frame}
\section{產出與結論}
\begin{frame}
\frametitle{Testing}
拿模型做回測\\
\begin{center}
\begin{tabular}{ccc}
函數$\backslash$ 實際 & 冬春& 夏秋\\
冬春 &0.9628 &0.1228 \\
夏秋 &0.0372 &0.8772 \\
\end{tabular}
\begin{tabular}{ccc}
函數$\backslash$ 實際 & 冬春& 夏秋\\
冬春 &1035 &137 \\
夏秋 &40 &979 \\
\end{tabular}
\end{center}
\begin{center}
\includegraphics[width=3in]{6.jpeg}
\end{center}
\end{frame}

\begin{frame}
\frametitle{Testing: 經過平滑後}
拿模型做回測\\
\begin{center}
\begin{tabular}{ccc}
函數$\backslash$ 實際 & 冬春& 夏秋\\
冬春 &0.974 &0.129 \\
夏秋 &0.026 &0.871 \\
\end{tabular}
\begin{tabular}{ccc}
函數$\backslash$ 實際 & 冬春& 夏秋\\
冬春 &1047 &144 \\
夏秋 &28 &972 \\
\end{tabular}
\end{center}
\begin{center}
\includegraphics[width=3in]{21.jpeg}
\end{center}
\end{frame}

\begin{frame}
\frametitle{註解頁}
\hspace*{4mm} 拿著做好的決策函數回過頭去測試2010到2015,發現到"冬春"以及"夏秋"交界處很容易分錯,且有些點不連續,因此我們使用一些小函數處理一些不連續的點。觀察adaboost的結果,發現它在交界處並沒有較明顯的特徵(如值突然貼近0或非常遠離0),因此我針對每一個該被分成夏天的點,如果這個點的左邊以及右邊都是冬天,那麼就變成夏天吧;針對該被分成冬天的點也一樣。當然這樣沒辦法讓不連續完全消失,但可以讓整個圖變得更好讀。
\end{frame}
\begin{frame}
\frametitle{檢測2016到2018}
\begin{center}
\includegraphics[width=3in]{8.jpeg}
\end{center}
\end{frame}
\begin{frame}
\frametitle{註解頁}
\hspace*{4mm} 我們拿2010到2015的資料為基準,觀察2016到2018的天氣型態,進而定論是否需要進行曆法調整,資料顯示曆法以及我們的決策函數所訂出來的"冬春"及"夏秋"相去不遠,決策函數的"夏秋"時間較長,但也長不過7天。
\end{frame}
\begin{frame}
\frametitle{下一步}
\emph{1. } 取得過去資料\\
\emph{2. } 等待未來資訊\\
\begin{center}
\includegraphics[width=2in]{9.jpg}
\end{center}
\end{frame}
\begin{frame}
\frametitle{註解頁}
\hspace*{4mm} 或許是因為訓練決策函數的資料點以及我們觀察的年份相去不遠,所以天氣型態沒有太大的改變。若我們能取得更久以前的資料,用那些資料當基準來觀察現在的天氣型態或許能發現很大的差別。又或是將我們的決策函數留下,觀察未來天氣型態是如何變化。\\
\hspace*{4mm} 或者我們可以改變研究方式,因為我們想研究的是長期的氣候型態而非單純的天氣,可以使用moving average、分時間區塊(而非單純每日)來進行分析。
\end{frame}
\section{結尾}
\begin{frame}
\frametitle{QA}
%%QA pictureQA.jpg
\begin{center}
\includegraphics[width=2in]{QA.jpg}
\end{center}
\end{frame}
\begin{frame}
\frametitle{謝謝聆聽}
%%
\begin{center}
\includegraphics[width=4in]{137899.jpg}
\end{center}
\end{frame}

\end{document}
